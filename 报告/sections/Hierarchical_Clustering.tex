\subsection{层次聚类介绍}
在最开始,每个数据点作为单独的群集开始,然后合并相邻的群集以形成树状结构。
简单来说,这可以被看作是自下而上的聚类方法。
层次聚类是一种强大的无监督学习算法,它通过创建数据的层次结构来对相似的数据进行分组。
该算法的优点在于能够帮助可视化和理解数据间的相似性。
此外,通过层次结构可以创建更详细的分组,并理解群组间的关系和层次结构。 \par

它主要应用于社交网络分析、客户细分、地理信息系统等领域。例如,在生物学中,它可以用于理解基因分析、物种分类和进化关系等。
总之,由于这种算法提供了基于数据相似性的分析,因此它对于识别数据的结构和模式、提取信息非常有用。 \par

通过测量数据之间的距离和聚类合并过程,可以获得有价值的信息。\par

\subsection{使用层次聚类对图像进行识别的效果分析}
\begin{figure}[H]
    \centering
    \subfigure[matiz blue]{
        \includegraphics[width=3cm]{../dataset_graph/dataset_1.jpg}
        \label{matiz blue}
    }\quad
    \subfigure[tiggo black]{
        \includegraphics[width=3cm]{../dataset_graph/dataset_2.jpg}
        \label{tiggo black}
  }\end{figure}
    
