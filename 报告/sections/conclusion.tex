本文基于传统方法和深度学习方法对图像进行识别,传统方法中,层次聚类找到了图像分类的主成分,根据主成分绘制出了图像分类的层次谱系图。
但是传统方法的准确率受到了很多因素的影响,比如数据集的大小,数据集的质量,数据集的分布等等。
在AdaBoost算法的分类中,随着boost迭代数的增加,识别准确率快速提高至0.38左右,但当迭代数继续增加时,准确率反而下降,最终维持在0.36左右不断震荡,这是因为迭代数过多导致了过拟合。 \par

而深度学习方法中,YOLO模型的准确率最高,在10代以内正确率就能超过0.90,在多次迭代之后达到了0.95以上,而且YOLO模型的准确率受到过拟合的影响较小。
除此之外,AlexNet模型在大约10代后正确率就超过0.50,在之后的40代的训练中准确度不断提高,最终维持在0.6左右。 \par

综上深度学习方法的准确率较高,且稳定性较好,但是传统方法仍然有其独特的优势,比如层次聚类可以找到图像分类的主成分,这对于图像分类的分析有着重要的意义。 \par