随着城市化进程的加速,车辆数量激增,交通管理难度加大。
为了改善城市交通状况,保障出行安全,开发自动车辆识别系统势在必行。
开发一个车辆识别系统可以提高交通管理效率、提升交通安全性。同时,车辆识别模块可以应用在诸多方面的问题上。 \par

有很多CV的方法可以进行车辆识别,这里主要可以分为传统方法和神经网络方法。
传统主要包括支持向量机、马尔可夫网络等。其特点是图像识别使用的模型相对简单、包含的参数数量有限。
同时,传统方法可能很依靠数据集的选取,在不同的数据集上可能有截然不同的表现。 \par

在CV领域,2010年出现了基于深度神经网络的模型。例如AlexNet,便是基于卷积神经网络的模型。
AlexNet具有高于传统CV方法的识别准确度\cite{AlexNet}。
之后又出现了ResNet\cite{ResNet}等方法,进一步提升其识别的准确度。
近些年,又出现了YOLO模型。
YOLO是一种划时代的单阶段目标检测算法。YOLO使用单次前馈网络即可完成检测,检测速度极快;
整图预测充分利用全局信息,检测精度高,因此被广泛使用\cite{YOLO}。
另外还有,Transformer模型,其是一种采用自注意力机制的深度学习模型,这一机制可以按输入数据各部分重要性的不同而分配不同的权重。
通过借鉴Transformer的设计思想,Google设计出ViT模型,也是一种识别准确度颇高的模型,且具有很强开创性的模型\cite{Vision_Transformers}。
在此基础上Facebook开发出LeViT模型,是其进一步分演进和发展\cite{LeViT} \par

本文将通过对比传统模型和神经网络模型,来比较分析不同模型的优劣,并分析其中的原理。